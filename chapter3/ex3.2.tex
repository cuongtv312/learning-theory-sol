\textbf{(Decision List).}
A decision list is a function of the following form.  
Let $\{i_1, \ldots, i_d\}$ be a permutation of $\{1, 2, \ldots, d\}$, and let 
$a_i, b_i \in \{0,1\}$ for $i = 1, \ldots, d+1$. Then the function $f(x)$ is defined as:
\[
f(x) =
\begin{cases}
b_1, & \text{if } x_{i_1} = a_1 \\
b_2, & \text{else if } x_{i_2} = a_2 \\
\vdots & \\
b_d, & \text{else if } x_{i_d} = a_d \\
b_{d+1}, & \text{otherwise.}
\end{cases}
\]

\bigskip
\textbf{Learning algorithm:}

\noindent
\textit{Input:} Training sample 
$S_n = \{(x_1, y_1), (x_2, y_2), \ldots, (x_n, y_n)\}$ 
where $x_i \in \{0,1\}^d$ and $y_i \in \{0,1\}$.  

\noindent
\textit{Output:} A decision list 
$D = [(i_1,a_1, b_1), (i_2,a_2, b_2), \ldots, (i_d,a_d, b_d), b_{d+1}]$.

\begin{enumerate}
\item Initialize $D = []$ and $S' = S_n$.
\item While $S'$ is not empty and $|D| < d$:
    \begin{enumerate}
    \item For each feature index $i \in \{1, \ldots, d\}$ and value $a \in \{0,1\}$,  
    check if all examples in $S'$ that satisfy $x_i = a$ have the same label $b$.
    \item If such $(i,a,b)$ exists, add $(i,a,b)$ to $D$.
    \item Remove from $S'$ all examples that satisfy the conditions.
    \end{enumerate}
\item If $S'$ is empty, set the default value $b_{d+1}$ to any remaining majority label.
\item Return $D$.
\end{enumerate}

\bigskip
\textbf{Running time analysis:}  
For each of at most $d$ iterations, the algorithm checks at most $2d$ literals 
($x_i = 0$ or $x_i = 1$), and for each literal scans up to $n$ examples.  
Hence, the total time complexity is
\[
O(n d^2)
\]

\bigskip
\textbf{Correctness:}  
If the target function is representable as a decision list,  
then at each iteration there exists at least one literal $(x_i = a)$ 
whose matching examples all share the same label.  
Thus, the algorithm will find one such pair $(a_i, b_i)$ per iteration and 
remove covered examples, leading to a perfect hypothesis on the training data:
\[
\widehat{\mathrm{err}}(\hat{f}) = 0.
\]

\bigskip
\textbf{PAC learnability:}  
The total number of distinct decision lists is at most 
\[
N = d! \cdot 2^{2d+1}.
\]
Therefore,
\[
\ln N = \ln(d!) + (2d+1)\ln 2 = O(d \log d),
\]
  

Hence, the class of decision lists is PAC-learnable because:
\begin{enumerate}
    \item the algorithm runs in polynomial time in $n$ and $d$,
    \item the hypothesis space size is polynomially bounded in $d$ and with theorem 3.6.
\end{enumerate}
Therefore, the Decision List concept class is \textbf{computationally efficiently PAC-learnable.}